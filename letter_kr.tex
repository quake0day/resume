%%%%%%%%%%%%%%%%%%%%%%%%%%%%%%%%%%%%%%%%%
% Plain Cover Letter
% LaTeX Template
% Version 1.0 (28/5/13)
%
% This template has been downloaded from:
% http://www.LaTeXTemplates.com
%
% Original author:
% Rensselaer Polytechnic Institute 
% http://www.rpi.edu/dept/arc/training/latex/resumes/
%
% License:
% CC BY-NC-SA 3.0 (http://creativecommons.org/licenses/by-nc-sa/3.0/)
%
%%%%%%%%%%%%%%%%%%%%%%%%%%%%%%%%%%%%%%%%%

%----------------------------------------------------------------------------------------
%	PACKAGES AND OTHER DOCUMENT CONFIGURATIONS
%----------------------------------------------------------------------------------------

\documentclass[12pt]{letter} % Default font size of the document, change to 10pt to fit more text

%\usepackage{newcent} % Default font is the New Century Schoolbook PostScript font 
\usepackage{helvet} % Uncomment this (while commenting the above line) to use the Helvetica font

% Margins
\topmargin=-1in % Moves the top of the document 1 inch above the default
\textheight=8.5in % Total height of the text on the page before text goes on to the next page, this can be increased in a longer letter
\oddsidemargin=-10pt % Position of the left margin, can be negative or positive if you want more or less room
\textwidth=6.5in % Total width of the text, increase this if the left margin was decreased and vice-versa

\let\raggedleft\raggedright % Pushes the date (at the top) to the left, comment this line to have the date on the right

\begin{document}

%----------------------------------------------------------------------------------------
%	ADDRESSEE SECTION
%----------------------------------------------------------------------------------------

\begin{letter}{} 

%----------------------------------------------------------------------------------------
%	YOUR NAME & ADDRESS SECTION
%----------------------------------------------------------------------------------------


\begin{center}
\Large\bf Recommendation Letter % Your name
\vspace{20pt} \hrule height 1pt % If you would like a horizontal line separating the name from the address, uncomment the line to the left of this text
\end{center} 

\signature{Kui Ren\\
Associate Professor, IEEE Fellow\\
Department of Computer Science and Engineering\\
University at Buffalo\\
Buffalo, NY\\
Tel: (716) 645-1587\\
Email:kuiren@buffalo.edu} % Your name for the signature at the bottom

%----------------------------------------------------------------------------------------
%	LETTER CONTENT SECTION
%----------------------------------------------------------------------------------------

\opening{Dear Faculty Search Committe,} 
 
As the advisor of Mr. Si Chen, one of my promising students in the Ubiquitous Security and Privacy Research Laboratory (UbiSeC Lab), it is my pleasure to write this letter of recommendation to support his tenure track faculty position application in your department.

Since I became Si's advisor and instructed his research in 2012, I have been continuously impressed by his excellent research ability as well as his self-motivated attitude. My long  with him leads me to be very familiar with his academic strengths and great personality.
%acquaintance
He first impressed me with his self-motivated quick learning and intelligent programming skills. He obtained the master degree from Electrical Engineering department at University at Buffalo before he joint our UbiSeC Lab. I inquired about his feedbacks from my colleagues including his previous academic advisor Dr. Tommaso Melodia. He laid a broad and solid foundation in both electrical engineering and computer science and showed his enthusiasms in becoming a researcher in the field of computer science early on. His excellent academic performance warranted his admission to our department for his further education as a PhD candidate in 2012. After that, I have assigned him to lead a keyless secure acoustic communication project where he built a purely software-based system to secure smartphone short-range communication without key agreement phase and it is well suited for legacy mobile devices. He self-learnt how to program on android platform, studied state-of-the-art audio coding scheme, grasped the usage of Django framework and finally integrated them in a prototype system. In fact, his mobile application is the first to demonstrate secure alternative NFC chiefly on friendly jamming technique for acoustic short-range communication. This alternative NFC technique provides NFC-like functionalities commercial smartphone applications, and enables much stronger security guarantees but require less strict hardware support. Under his leadership, this prototype system is selected in the finalist (top 10 projects) of ACM Mobicom mobile app competition. 

Si then showed his critical thinking and quantitative ability in his research studies of our mobile sensing project. Si's first major research result was in designing and implementing a crowdsourcing system utilizing sensor-rich video data from mobile users for indoor floor plan reconstruction. This is a difficult problem as it needs strong ability in both system architecture and algorithm design. Si first analyzed the problem and then exploited the sequential relationship between each consecutive frame abstracted from the video to improve system performance. He then built a real system with both mobile front-end and cloud back-end and conduct comprehensive experiments in a real-world scenario. The experiments results demonstrated that his techniques achieved a significant improvement of accuracy compared with other state-of-the-art crowdsourcing floor plan reconstruction systems. Going beyond 2D, he further proposed IndoorCrowd2D, a smartphone-empowered crowdsourcing system for indoor scene reconstruction. He came up with a divide and conquer method for solving the problem. Then, he designed trackable models to represent indoor space, implemented them independently and demonstrated its advantages by comparing with other systems. Especially, Si's work is the first to propose, design and implement a smartphone-based crowdsourcing system that explores the power of untrained individuals to generate building interior views at scale. It breaks away from established approaches to reconstruct indoor scenes, and explores an advanced architecture based on crowdsourcing and mobile-sensing. Moreover, the system achieves a significant improvement of accuracy compared with other indoor scene reconstruction systems, according to a long-term real-world experiment on 30 volunteers. More importantly, the system he built is readily deployable in real-world scenarios. It is also expected to extend existing online map services to the indoor environments at an unprecedented scale, which is currently cost prohibitive. In a long view, this system is serve an important stepping stone towards economically-viable massive indoor 3D model reconstruction. The results of the two research projects have been published at IEEE International Conference on Distributed Computing Systems (ICDCS'15) and ACM conference on Embedded Networked Sensor Systems (SenSys'15), which are highly selective research venues on systems issues of broadly-defined sensors and sensor-enabled smart systems.

Si has continually impressed me with his brilliant problem-solving ability, solid programming ability, independent thinking and strong mathematics background. Si also participates in a location-based social networks (LBSNs) project. He developed an automated user location tracking system, which can accurately geo-locate any target in LBSN. The outcome of this project serves as a critical security reminder of the current LBSNs pertaining to a vast number of users. This work has been published in ACM MobiHoc'14.

Additionally, Si has a strong desire and great enthusiasm to pursue new knowledge. He is constantly reading to keep track with the latest trends in academia. He keeps exploring untapped challenges brought by emerging technologies and investigating novel ideas and techniques to tackle practical problems for today's and tomorrow's cyber physical systems. He also analyzes the contributions of new papers carefully, writes programs to verify how good they are, and discusses with me his new ideas. He believes researchers in computer science field must be continuous to learn, grow and experience things in life to adapt the rapid changing technology. As his academic advisor, I worked closely with him on making his study plans, improving his academic performance, and shaping his career goals, and thereby, I clearly understand that his career goal is becoming a faculty in the field of cybersecurity. I believe his self-motivated attitude will be very beneficial to his future career and research.

Besides his academic ability, Si also has the talent in teaching courses and mentoring fresh or junior level students. He told me that the teacher is a sacred occupation and it is amazing to let the knowledge be passed from one to many by good communication skill, and thereby, he enjoys sharing what he knows with more students. He explains complicate knowledge by simple language, which is the premise to obtain good teaching results. Students who have had attended his classes and seminars can always give positive feedbacks about him.

Last but not least, Si has lovely personality. His classmates and labmates also considered Si warm-hearted and considerate to others. Si believes that strong teamwork is as significant as the ability of individuals. Besides being a researcher, he is also a great leader and a helpful partner in our lab. People who have collaborated with him think him has a stable personality and high sense of responsibility. 

Overall, I am very excited with Si's accomplishments and abilities. Compared with my other talented young students I have had the fortune to mentor, Si is among the most outstanding. He is a mature and energetic with strong career ambition. I am pleased that he will continue his pursuit for high-quality research in academia and I expect great research achievements from him long into the future. 






\closing{Sincerely yours,}


%\encl{Curriculum vitae, employment form} % List your enclosed documents here, comment this out to get rid of the "encl:"

%----------------------------------------------------------------------------------------

\end{letter}

\end{document}