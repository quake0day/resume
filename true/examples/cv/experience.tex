\cvsection{Projects}
\begin{cventries}
  \cventry
    {This project aims at developing a \textbf{highly secure alternative NFC system}.}
    {AcousAuth: An acoustic-based mobile application for user authentication}
    {Project Leader}
    {Jan. 2013 - Aug. 2013}
    {
      \begin{cvitems}
      \item \textbf{GitHub:} \href{https://github.com/quake0day/Jigglypuff}{\textbf{\underline{https://github.com/quake0day/Jigglypuff}}} \ \ \textbf{Vimeo:} \href{http://vimeo.com/77708077}{\textbf{\underline{http://vimeo.com/77708077}}}
      \item \textbf{Techniques: Python, C++, Android, jQuery, HTML5, Web.py, Django, MySQL, CSS(Bootstrap UI)} 
\item Proposed an alternative NFC technique which provides NFC-like functionalities commercial smartphone applications, and enables much stronger security guarantees but requires less strict hardware support. 
\item Designed a smartphone empowered system for personal authentication featuring
a seamless, faster, easier and safer authentication process without the need of special
infrastructure.
\item Shortlisted in the 19th Annual International Conference on Mobile Computing and Networking (ACM Mobicom'13) App Competition \href{http://www.sigmobile.org/mobicom/2013/app_finalists.html}{\textbf{\underline{(Top 10)}}}.
      \end{cvitems}
    }
  \cventry
    {This project is \textbf{the first to propose, design and implement} a smartphone-based crowdsourcing system that explores the power of untrained individuals to generate building interior views at scale.}
    {IndoorCrowd: Mapping of indoor building structures through mobile devices}
    {Project Leader}
    {Jun. 2013 - Feb. 2015}
    {
      \begin{cvitems}
            \item \textbf{GitHub:} \href{https://github.com/quake0day/Dragonite}{\textbf{\underline{https://github.com/quake0day/Dragonite}}}
      \item \textbf{Techniques: Python, Spark, Android, C++ (OpenCV), Web.py, jQuery, Matlab, MySQL, WebGL, CSS(Bootstrap UI)} 
\item Proposed a low-cost crowdsource-based method to reconstruct indoor floor plan that by utilizing sensor-rich video data from mobile users.
\item Innovatively exploited the sequential relationship between consecutive
frames to improve system performance.
\item Achieved a significant improvement of accuracy compared with other indoor scene reconstruction systems, according to a long-term real-world experiment on 30 volunteers. 
%\item Readily deployable in real-world scenarios. It is also expected to extend existing online map services (e.g. Google Map) to the indoor environments at an unprecedented scale, which is currently cost prohibitive. 
\item Served as an important stepping stone towards economically-viable massive indoor 3D model reconstruction. 
      \end{cvitems}
    }
  \cventry
    {This project serves as a \textbf{critical security reminder} of the current location based social networks (LBSNs) pertaining to a vast number of users. }
    {FreeTrack: Tracking mobile social network users}
    {Core Member}
    {Jan. 2013 - May. 2013}
    {
      \begin{cvitems}
                  \item \textbf{GitHub:} \href{https://github.com/kkspeed/FreeTrack}{\textbf{\underline{https://github.com/kkspeed/FreeTrack}}}
      \item \textbf{Techniques: Clojure, Android (App Dev, Disassembly with Smali, MonkeyRunner)} 
\item Identified severe location privacy leaks from popular location based social networks (e.g. Momo,
Skout and Wechat) that allows non-priviledged attacker to effectively pinpoint users' locations and even
performed long-term tracking to reveal identity. 
\item Developed an automated user location tracking system
and tested it on the these LBSNs. 
\item Demonstrated its effectiveness and efficiency via a 3-week real-world
experiment with 30 volunteers.
\item The evaluation results showed that this system can geo-locate a target with high
accuracy and can readily recover users' Top 5 locations. 
\item Proposed using grid reference system
and location classification to mitigate the attacks.
      \end{cvitems} 
    }
  \cventry
    {The aim of this project is to \textbf{accomplish an online simulation system based on NS-2} network simulator to simulate advanced high frequency (HF) network.}
    {Ground Wave Simulator}
    {Project Leader}
    {Jan. 2012 - Jan. 2013}
    {
      \begin{cvitems}
                        \item \textbf{GitHub:} Front-end \href{https://github.com/quake0day/Torterra}{\textbf{\underline{https://github.com/quake0day/Torterra}}}, Backend \href{https://github.com/quake0day/Groudon}{\textbf{\underline{https://github.com/quake0day/Groudon}}}
      \item \textbf{Techniques: Python, C++, Perl(Catalyst, DBIx), Tcl/Tk, MySQL, HTML, CSS(Bootstrap UI)} 
\item Studied models of HF channel characteristics, waveforms, protocols, and typical traffic loads.
\item Designed an online simulation system to calculate the electric field strength and basic path loss in the real-world environment.
\item Implemented an online ground wave simulation system based on NS-2 open source network simulator, GRWAVE and VOACAP Software to simulate advanced high frequency (HF) network.      \end{cvitems}
    }
  \cventry
    {This project aims at \textbf{developing a greenhouse management system} using wireless sensor network.}
    {Intellectualized Greenhouse Measuring \& Control System}
    {Project Leader}
    {May. 2009 - May. 2010}
    {
      \begin{cvitems}
            \item \textbf{Techniques: Python, C, Python, PHP, MySQL, Javascript} 
		\item{Designed a mathematical model specialized for simulating greenhouse environment.}
		\item{Used CC2430 wireless node and Zigbee stack (Z-stack) to measure and control a greenhouse model's humidity and temperature.}
		\item{Used Python, PHP, Javascript (jQuery), MySQL and C to create a realtime B/S System.}
		\item{Designed a PCB with controllers that can use CC2430 with computer to remote control the greenhouse model.}
		\item{Implemented PHP reflection mechanism to create a plugin system for further system enhancement.}
      \end{cvitems}
      %\begin{cvsubentries}
      %  \cvsubentry{}{KNOX(Solution for Enterprise Mobile Security) Penetration Testing}{Sep. 2013}{}
      %  \cvsubentry{}{Smart TV Penetration Testing}{Mar. 2011 - Oct. 2011}{}
      %\end{cvsubentries}
    }
\end{cventries}


%\end{itemize}